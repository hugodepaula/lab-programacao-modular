%%%%%%%%%%%%%%%%%%%%%%%%%%%%%%%%%%%%%%%
% Cronograma de aula
%%%%%%%%%%%%%%%%%%%%%%%%%%%%%%%%%%%%%%%

\documentclass[11pt,brazil, a4paper, fullpage]{article}


\newcommand\UNIVERSIDADE{PONTIFÍCIA  UNIVERSIDADE  CATÓLICA  DE  MINAS  GERAIS}
\newcommand\UNIDADE{Praça da Liberdade}
\newcommand\INSTITUTO{Inst. de Ciências Exatas e Informática}
\newcommand\CURSO{Ciência da Computação}
\newcommand\PROFESSOR{Hugo de Paula}
\newcommand\EMAILPROFESSOR{hugo@pucminas.br}
\newcommand\DEPARTAMENTO{Departamento de Ciência da Computação}

\newcommand\DISCIPLINA{Laboratório de Programação Modular}
\newcommand\ANO{2024}
\newcommand\SEMESTRE{1}
\newcommand\TURNO{manhã}
\newcommand\PERIODO{3}

\def\year{\ANO}

\input{infos/some-definitions}
\input{infos/calpreambulo}

\newcommand{\DIAUM}{\Sexta}
\newcommand{\DIADOIS}{\Sexta}


\begin{document}

	\selectlanguage{brazil}

	\logoPUC{infos/puclogo_small_bw}
	\dadosDisciplina{\DISCIPLINA}{\CURSO}{\TURNO}{\ANO}{\SEMESTRE}{\PROFESSOR}{\EMAILPROFESSOR}

	%\dadosAvancadosDisciplina{04}{00}{04}{80}{04}{infos/objetivos.tex}{infos/ementa.tex}{3 $\times$ 25}{25}{infos/avaliacao.tex}{infos/biblio.tex}

	\calVisualColorido{1}{\DIAUM}{\DIADOIS}{\ANO}{07-07} % semestre, dias de aula (DOIS vezes por semana)
	%\calVisualColorido{2}{\DIAUM}{\DIADOIS}{\ANO}{12-21} % semestre, dias de aula (DOIS vezes por semana)

\section{Especificação}

\subsection{Sprint 1}

\begin{center}
    \setlength{\fboxsep}{10pt}
    \fbox{%
        \parbox{.7\textwidth}{\textbf{Objetivos}: \\[1mm]
            modelo UML (agregação), classes de negócio e testes unitários, versão 1 do sistema.}%
    }
\end{center}


Um restaurante atende seus clientes alocando-os em mesas por meio de 'requisições por mesas'.

No momento, o restaurante possui 10 mesas: 4 com capacidade para 4 pessoas, 4 com capacidade para 6 pessoas e 2 com capacidade para 8 pessoas. O restaurante deve ser capaz de ampliar a sua capacidade com o menor esforço possível.

O cliente, ao chegar, deve declarar quantas pessoas comerão no restaurante, gerando assim uma requisição por mesa.

Tão logo quanto possível deverá ser alocada uma mesa adequada para ele. Se não houver mesa livre, o cliente entra numa fila de espera. As requisições precisam registrar as datas e horas de entrada e saída do cliente.




\subsection{Sprint 2}


\begin{center}
    \setlength{\fboxsep}{10pt}
    \fbox{%
        \parbox{.7\textwidth}{\textbf{Objetivos}: \\[1mm]
            atualizar projeto UML, novas classes e testes, novas regras de negócio e persistência/spring.}%
    }
\end{center}


O restaurante oferece aos clientes um cardápio simplificado com 6 pratos e 5 opções de bebidas como por exemplo: \\


\begin{tabular}{ll}
\begin{minipage}{.45\textwidth}
\begin{itemize}
    \item Moqueca de Palmito
    \item Falafel Assado
    \item Salada Primavera com Macarrão Konjac
    \item Escondidinho de Inhame
    \item Strogonoff de Cogumelos
    \item Caçarola de carne com legumes
\end{itemize}
\end{minipage} &

\begin{minipage}{.45\textwidth}
\begin{itemize}
    \item Água
    \item Suco
    \item Refrigerante
    \item Cerveja
    \item Taça de vinho
\end{itemize}
\end{minipage} \\
\end{tabular} \\



Cada mesa, enquanto está sendo servida, pode fazer um pedido com quantos pratos e bebidas desejar. Novos produtos podem ser adicionados ao pedido no decorrer do atendimento.

Ao encerrar o pedido, a conta incluirá uma taxa de serviço de 10\% e exibirá tanto o valor total como o valor a ser dividido igualmente entre os ocupantes.


\subsection{Sprint 3}



\begin{center}
    \setlength{\fboxsep}{10pt}
    \fbox{%
        \parbox{.7\textwidth}{\textbf{Objetivos}: \\[1mm]
            atualizar projeto UML, novas classes e testes com herança e/ou composição com interface.}%
    }
\end{center}


Cada conta pode ser paga por diferentes métodos: dinheiro, Pix, débito ou crédito. O que muda para cada método são o prazo para receber o valor e a taxa que o restaurante paga ao banco por este recebimento. Além disso, para pagamento em Pix, deve ser armazenado o nome do emitente do Pix. No caso de débito, deve ser armazenado o nome do banco de origem do débito. Finalmente, no caso de cartão de crédito, deve ser armazenado a bandeira do cartão (Visa, Mastercard, Elo, American Express).

\begin{itemize}
    \item \textbf{dinheiro}: prazo 0 dias, desconto 0\%
    \item \textbf{pix}: prazo 0 dias, desconto de 1,45\%, limitado a R\$10
    \item \textbf{débito}: prazo 14 dias, desconto de 1,4\%
    \item \textbf{crédito}: prazo 30 dias, desconto de 3,1\%
\end{itemize}

O restaurante precisa saber quanto vendeu em um dia e quanto receberá em datas futuras a serem determinadas pela gerência.

Pedidos para delivery não incluem taxa de serviço, mas incluem taxa de entrega de acordo com a distância: Estes pedidos não precisam especificar o valor individual a ser pago em uma conta.


\subsection{Sprint 4}


\begin{center}
    \setlength{\fboxsep}{10pt}
    \fbox{%
        \parbox{.7\textwidth}{\textbf{Objetivos}: \\[1mm]
            atualizar projeto UML, novos requisitos para incorporar tratamentos de erros, exceções e padrões de projeto.}%
    }
\end{center}

Esse trabalho deve usar padrões de projeto, como por exemplo: \textbf{Singleton}, \textbf{Decorator}, \textbf{Factory} e \textbf{Observer}.


\subsection{Critérios de correção}

Será avaliado:
\begin{itemize}
    \item a conformidade com os requisitos
    \item qualidade do código: correção e robustez
    \item modularização: alta coesão e baixo acoplamento, extensibilidade e adequação aos princípios SOLID
    \item utilização dos conceitos básicos e avançados de programação modular: tais como polimorfismo, collections, streams e tratamento de exceções.
    \item testes unitários baseados na JUnit.
\end{itemize}

\newpage

\section{Cronograma de realização de tarefas}

\begin{center}

\small
%\footnotesize
\begin{calendar}{3/15/\ANO}{15} % Semestre comeca no dia 1 de agosto, e dura por 21 semanas.
\setlength{\calboxdepth}{.3in}
\setlength{\calwidth}{0.95\textwidth}

%
%% configuracao da semana
\semanaSex
%
%% lista de Aulas
\caltexton{1}{Sprint 1 -- Diagrama de Classes na UML }
\caltextnext{Sprint 1 -- Implementação classes de negócio e testes unitários, versão 1 do sistema. \\
    \textbf{Apresentação da Sprint 1}}
\caltextnext{Aula de SpringBoot \\
Sprint 2 -- implementação dos novos requisitos.}
\caltextnext{Aula de Herança e polimorfismo \\
Sprint 2 -- implementação da persistência / spring.}
\caltextnext{Aula de Interfaces \\
    Sprint 2 -- implementação dos testes.}
\caltextnext{\textbf{Apresentação da Sprint 2}}
\caltextnext{Aula de Collections e Streams. \\
    Sprint 3 -- implementação de polimorfismo e interfaces.}
\caltextnext{Sprint 3 -- implementação de collections e streams.}
\caltextnext{Aula de Padrões de Projeto}
\caltextnext{\textbf{Apresentação da Sprint 3}}
\caltextnext{Aula de Eventos e interface gráfica \\
    Sprint 4 -- implementação dos padrões de projeto.}
\caltextnext{Sprint 3 -- implementação dos tratamentos de erro.}
%\caltextnext{Acompanhamento}
\caltextnext{\textbf{Apresentação da Sprint 4}}
\caltextnext{Reavaliação}
%
% feriados e avisos
% Se n?o quiser que os avisos sejam colocados no cronograma, basta comentar o comando input.
\input{infos/feriados}
%%\input{infos/avisos}
%

%\aviso{9/19/2022}{AULA ADICIONAL SEGUNDA-FEIRA, DIA 19/09, 10h40}
%\aviso{10/12/2022}{AULA ADICIONAL QUINTA-FEIRA, DIA 13/10, 8h50}
%\aviso{10/31/2022}{AULA ADICIONAL SEGUNDA-FEIRA, DIA 19/09, 10h40}


\end{calendar}
\end{center}

\end{document}
